% --------------------------------------------------------------
%                         Packages
% --------------------------------------------------------------

\usepackage{
	graphicx,
	amsmath,
	amssymb,
	mathtools,
	url,
	color,
	lmodern,
	setspace,
	tikz,
	ifthen,
	xparse,
	pgfplots
	}

% --------------------------------------------------------------
%                         Packages with settings
% --------------------------------------------------------------
\usepackage[utf8]{inputenc}
\usepackage[ruled,linesnumbered,noend]{algorithm2e}
\usepackage{hyperref}

\usepackage{natbib}
%\setcitestyle{authoryear,round,citesep={;},aysep={,},yysep={;}}
\setcitestyle{numbers,square,longnamesfirst}

\usepackage{paralist}
\renewenvironment{enumerate}[1]{\begin{compactenum}#1}{\end{compactenum}}

\usetikzlibrary{shapes,arrows,trees,automata,positioning,matrix}
\pgfplotsset{compat=1.16}

% --------------------------------------------------------------
%                         Preferences settings
% --------------------------------------------------------------

% Appendix Settings
\renewcommand{\appendixname}{}

% CleverRef settings
\usepackage[capitalize,nameinlink]{cleveref}
\usepackage{crossreftools}
\pdfstringdefDisableCommands{%
    \let\Cref\crtCref
    \let\cref\crtcref
}
\newcommand{\creflastconjunction}{, and\nobreakspace}
\hypersetup{%
    bookmarksnumbered, bookmarksopen=true, bookmarksopenlevel=1,%
}

\crefname{subsection}{Subsection}{Subsections}
\crefname{lemma}{Lemma}{Lemmas}
\crefname{corollary}{Corollary}{Corollaries}
\crefname{theorem}{Theorem}{Theorems}

% --------------------------------------------------------------
%                         Format settings
% --------------------------------------------------------------

% Equations
\newcommand{\manualendproof}{\hfill\qedsymbol\\[2mm]}

% All equation numbers same size
\makeatletter
\renewcommand{\maketag@@@}[1]{\hbox{\m@th\normalsize\normalfont#1}}%
\makeatother

% Grey out equation numbers
\makeatletter
\let\reftagform@=\tagform@
\def\tagform@#1{\maketag@@@{\ignorespaces\textcolor{gray}{(#1)}\unskip\@@italiccorr}}
\renewcommand{\eqref}[1]{\textup{\reftagform@{\ref{#1}}}}
\makeatother

% --------------------------------------------------------------
%                         Syntactic macros
% --------------------------------------------------------------

% Blackboard font
\providecommand{\AA}{\mathbb{A}}
\newcommand{\BB}{\mathbb{B}}
\newcommand{\CC}{\mathbbm{C}}
\newcommand{\DD}{\mathbbm{D}}
\newcommand{\EE}{\mathbb{E}}
\newcommand{\FF}{\mathbb{F}}
\newcommand{\GG}{\mathbb{G}}
\newcommand{\HH}{\mathbb{H}}
\newcommand{\II}{\mathbb{I}}
\newcommand{\JJ}{\mathbb{J}}
\newcommand{\KK}{\mathbb{K}}
\newcommand{\LL}{\mathbb{L}}
\newcommand{\MM}{\mathbb{M}}
\newcommand{\NN}{\mathbb{N}}
\newcommand{\OO}{\mathbb{O}}
\newcommand{\PP}{\mathbb{P}}
\newcommand{\QQ}{\mathbb{Q}}
\newcommand{\RR}{\mathbb{R}}
\renewcommand{\SS}{\mathbb{S}}
\newcommand{\TT}{\mathbb{T}}
\newcommand{\UU}{\mathbb{U}}
\newcommand{\VV}{\mathbb{V}}
\newcommand{\WW}{\mathbb{W}}
\newcommand{\XX}{\mathbb{X}}
\newcommand{\YY}{\mathbb{Y}}
\newcommand{\ZZ}{\mathbb{Z}}

\newcommand{\One}{\mathbbm{1}}
\newcommand{\Two}{\mathbbm{2}}
\newcommand{\Three}{\mathbbm{3}}
\newcommand{\Four}{\mathbbm{4}}
\newcommand{\Five}{\mathbbm{5}}
\newcommand{\Six}{\mathbbm{6}}
\newcommand{\Seven}{\mathbbm{7}}
\newcommand{\Eight}{\mathbbm{8}}
\newcommand{\Nine}{\mathbbm{9}}
\newcommand{\Zero}{\mathbbm{0}}

% Calligraphy Font
\newcommand{\Aa}{\mathcal{A}}
\newcommand{\Bb}{\mathcal{B}}
\newcommand{\Cc}{\mathcal{C}}
\newcommand{\Dd}{\mathcal{D}}
\newcommand{\Ee}{\mathcal{E}}
\newcommand{\Ff}{\mathcal{F}}
\newcommand{\Gg}{\mathcal{G}}
\newcommand{\Hh}{\mathcal{H}}
\newcommand{\Ii}{\mathcal{I}}
\newcommand{\Jj}{\mathcal{J}}
\newcommand{\Kk}{\mathcal{K}}
\newcommand{\Ll}{\mathcal{L}}
\newcommand{\Mm}{\mathcal{M}}
\newcommand{\Nn}{\mathcal{N}}
\newcommand{\Oo}{\mathcal{O}}
\newcommand{\Pp}{\mathcal{P}}
\newcommand{\Qq}{\mathcal{Q}}
\newcommand{\Rr}{\mathcal{R}}
\newcommand{\Ss}{\mathcal{S}}
\newcommand{\Tt}{\mathcal{T}}
\newcommand{\Uu}{\mathcal{U}}
\newcommand{\Vv}{\mathcal{V}}
\newcommand{\Ww}{\mathcal{W}}
\newcommand{\Xx}{\mathcal{X}}
\newcommand{\Yy}{\mathcal{Y}}
\newcommand{\Zz}{\mathcal{Z}}

% Bold numbers
\newcommand{\one}{\mathbf{1}}
\newcommand{\two}{\mathbf{2}}
\newcommand{\three}{\mathbf{3}}
\newcommand{\four}{\mathbf{4}}
\newcommand{\five}{\mathbf{5}}
\newcommand{\six}{\mathbf{6}}
\newcommand{\seven}{\mathbf{7}}
\newcommand{\eight}{\mathbf{8}}
\newcommand{\nine}{\mathbf{9}}
\newcommand{\zero}{\mathbf{0}}

% Nicer greek letters
\newcommand{\eps}{\varepsilon}
\newcommand{\gam}{\vargamma}

% --------------------------------------------------------------
%                         Quality of Life Macros
% --------------------------------------------------------------

% Math Environments
% redefines \[ ...\]  mathmode to use AMS "align" environment, which is superior
\def\[#1\]{\begin{equation}\begin{aligned}#1\end{aligned}\end{equation}}
\def\*[#1\]{\begin{equation*}\begin{aligned}#1\end{aligned}\end{equation*}}
\def\s*[#1\s]{\small\begin{align*}#1\end{align*}\normalsize}

% Bracket Shorthand
\newcommand{\lcr}[3]{\left #1 #2 \right #3} % autosize
\newcommand{\lcrx}[4][{-1}]{ 
	\IfEq{#1}{-1}{\left #2 {{{{#3}}}} \right #4}{
   	\IfEq{#1}{0}{#2 {{{{#3}}}} #4}{
	\IfEq{#1}{1}{\bigl #2 {{{{#3}}}} \bigr #4}{
	\IfEq{#1}{2}{\Bigl #2 {{{{#3}}}} \Bigr #4}{
	\IfEq{#1}{3}{\biggl #2 {{{{#3}}}} \biggr #4}{
	\IfEq{#1}{4}{\Biggl #2 {{{{#3}}}} \Biggr #4}{
    \GenericWarning{"4th argument to lcrx must be -1, 0, 1, 2, 3, or 4"}
    }}}}}}} % specify size with {-1,...4} as optional argument

% Accents
\newcommand{\ol}{\overline}
\newcommand{\ul}{\underline}
\newcommand{\mb}{\mathbf}
\newcommand{\upper}[1]{^{(#1)}}

% --------------------------------------------------------------
%                         Semantic Macros
% --------------------------------------------------------------


% Probability operators
\newcommand{\KL}[2]{\mathrm{KL}\rbra{#1 \ \Vert \ #2}} % Kullback-Leibler divergence
\DeclareMathOperator*{\sdev}{StdDev} % standard deviation
\DeclareMathOperator{\E}{\EE} % Expectation
\DeclareMathOperator{\Var}{Var} % Variance
\DeclareMathOperator{\Prob}{\PP} % Probability
\DeclareMathOperator{\Cov}{Cov} % Covariance
\DeclareMathOperator{\Corr}{Corr} % Correlation
\DeclareMathOperator{\Bias}{Bias} % Bias
\DeclareMathOperator{\Mse}{MSE} % Mean square error
\DeclareMathOperator{\R}{R} % R squared
\newcommand{\prodmeas}[1]{^{\otimes #1}}
\newcommand{\suchthat}{\;\ifnum\currentgrouptype=16 \middle\fi|\;} % Vertical bar
\DeclareMathOperator{\indep}{\mathrel{\text{\scalebox{1.07}{$\perp\mkern-10mu\perp$}}}} % Independance.
\newcommand{\negsetdelim}{\vert}
\newcommand{\setdelim}{\ \vert \ }
\newcommand{\Bigsetdelim}{\ \Big\vert \ }
\newcommand{\Biggsetdelim}{\ \Bigg\vert \ }
\newcommand{\iid}{\text{i.i.d.}}
\newcommand{\IID}{\text{I.I.D.}}
\newcommand{\as}{\text{a.s.}}


% Words in math
\newcommand{\stT}{\ \text{s.t.}\ }
\newcommand{\andT}{\ \text{and}\ }
\newcommand{\orT}{\ \text{or}\ }
\newcommand{\whereT}{\ \text{where}\ }
\newcommand{\withT}{\ \text{with}\ }


% Special functions
\DeclareMathOperator{\linspan}{span}
\DeclareMathOperator{\logit}{logit}
\DeclareMathOperator{\sgn}{sgn}
\newcommand{\Ind}{\mathds 1} % indicator function
\def\multiset#1#2{\ensuremath{\left(\kern-.3em\left(\genfrac{}{}{0pt}{}{#1}{#2}\right)\kern-.3em\right)}}


% Limit-like operators
\DeclareMathOperator*{\argmin}{\arg\min} % argmin
\DeclareMathOperator*{\argmax}{\arg\max} % argmax
\DeclareMathOperator*{\esssup}{\text{ess}\sup} % essential supremum
\DeclareMathOperator*{\essinf}{\text{ess}\inf} % essential infemum

% Fixes to the standard operators to improve kerning
\DeclareMathOperator*{\newlim}{\mathrm{lim}\vphantom{\mathrm{infsup}}}
\DeclareMathOperator*{\newmin}{\mathrm{min}\vphantom{\mathrm{infsup}}}
\DeclareMathOperator*{\newmax}{\mathrm{max}\vphantom{\mathrm{infsup}}}
\DeclareMathOperator*{\newinf}{\mathrm{inf}\vphantom{\mathrm{infsup}}}
\DeclareMathOperator*{\newsup}{\mathrm{sup}\vphantom{\mathrm{infsup}}}
\renewcommand{\lim}{\newlim}
\renewcommand{\min}{\newmin}
\renewcommand{\max}{\newmax}
\renewcommand{\inf}{\newinf}
\renewcommand{\sup}{\newsup}


% Linear Algebra
\newcommand{\tr}{^\text{T}} % transpose
\newcommand{\trace}{\mathrm{Tr}} % trace
\newcommand{\adj}{^{\dag}} % adjoint
\DeclareMathOperator{\spec}{\sigma} % spectrum
\DeclareMathOperator{\diag}{diag} % diagonal
\DeclareMathOperator{\rank}{rank} % rank


% Calculus
\newcommand{\dee}{\mathrm{d}} % for integrals \int f(x) \dee x
\newcommand{\grad}[1]{\nabla_{#1}} % gradient
\newcommand{\laplace}{\Delta} % laplacian
\newcommand{\hess}{\grad^2} % hessian
\DeclareMathOperator*{\divergence}{div} % divergence
\newcommand{\jacobian}{J}
\newcommand{\conv}{*}


% Measure Theory
%\newcommand{\pdiff}[2]{\frac{\partial #1}{\partial #2}}
%\newcommand{\diff}[2]{\frac{\dee #1}{\dee #2}}
\DeclareDocumentCommand{\virtualDiff}{m m G{} G{}}{\frac{#1^{#4} \kern 1pt #3}{#1 \kern 1pt #2^{#4}}}
\DeclareDocumentCommand{\pdiff}{m G{} G{}}{\virtualDiff{\partial}{#1}{#2}{#3}} % Partial derivative \pdiff{denominator}{optional numerator}{optional exponant}
\DeclareDocumentCommand{\diff}{m G{} G{}}{\virtualDiff{\text d}{#1}{#2}{#3}} % Derivative
\DeclareMathOperator{\supp}{Support}


% Probability Theory
\newcommand{\distto}{\rightsquigarrow}
\newcommand{\disttosk}{\Rightarrow}
\newcommand{\universe}{\Omega}

\newcommand{\distas}{\sim}
\newcommand{\distiidas}{\stk{\distas}{iid}}
\newcommand{\distindas}{\stk{\distas}{ind}}

\newcommand{\pushfwdmeas}[2]{{{#1}_{\sharp} #2}}
\newcommand{\Law}{\Ll}
\newcommand{\cLaw}[1]{\Law^{#1}} % Conditional law
\newcommand{\kernels}{\Kk}


% Special Distributions
\newcommand{\poissondist}{\mathrm{Pois}}
\newcommand{\negbinomialdist}{\mathrm{NBin}}
\newcommand{\binomialdist}{\mathrm{Bin}}
\newcommand{\bernoullidist}{\mathrm{Ber}}
\newcommand{\betadist}{\mathrm{Beta}}
\newcommand{\geometricdist}{\mathrm{Geom}}
\newcommand{\hypergeometricdist}{\mathrm{HGeom}}
\newcommand{\exponentialdist}{\mathrm{Exp}}
\newcommand{\gammadist}{\mathrm{Gamma}}
\newcommand{\normaldist}{\mathrm{Nor}}
\newcommand{\lognormaldist}{\mathrm{LogN}}
\newcommand{\uniformdist}{\mathrm{Unif}}


% Brackets and bracket-like functions
\newcommand{\rbra}[2][{-1}]{\lcrx[#1] ( {#2} ) }
\newcommand{\cbra}[2][{-1}]{\lcrx[#1] \{ {#2} \} }
\newcommand{\sbra}[2][{-1}]{\lcrx[#1] [ {#2} ] }

\newcommand{\abs}[2][{-1}]{\lcrx[#1] \vert {#2} \vert }
\newcommand{\set}[2][{-1}]{\lcrx[#1] \{ {#2} \}}
\newcommand{\floor}[2][{-1}]{\lcrx \lfloor {#2} \rfloor}
\newcommand{\ceil}[2][{-1}]{\lcrx[#1] \lceil {#2} \rceil}
\newcommand{\norm}[2][{-1}]{\lcrx[#1] \Vert {#2} \Vert}
\newcommand{\inner}[3][{-1}]{\lcrx[#1] \langle {{#2},\ {#3}} \rangle}
\newcommand{\card}[2][{-1}]{\lcrx[#1] \vert {#2} \vert }


% Definition 
\DeclareMathOperator{\eqd}{\overset{d}{=}} % Equality in distribution.
\DeclareMathOperator{\convd}{\overset{d}{\longrightarrow}} % Convergence in distribution.
\DeclareMathOperator{\convp}{\overset{p}{\longrightarrow}} % Convergence in probability.
\DeclareMathOperator{\convas}{\overset{a.s.}{\longrightarrow}} % Convergence almost surely.

% Special Sets
\newcommand{\Nats}{\NN}
\newcommand{\NatsO}{\Nats\cup\set{0}}
\newcommand{\Ints}{\ZZ}
\newcommand{\Rationals}{\QQ}
\newcommand{\Rats}{\Rationals}
\newcommand{\Reals}{\RR}
\newcommand{\ExtReals}{\overline{\Reals}}
\newcommand{\PosInts}{\Ints_+}
\newcommand{\PosReals}{\Reals_+}
\newcommand{\Measures}{\mathcal{M}}
\newcommand{\ProbMeasures}{\mathcal{M}_1}

\newcommand{\range}[2][{1}]{
	\IfEq{#1}{1}{\sbra{#2}}{\sbra{#2}_{#1}}}
\newcommand{\rangeO}[2][{0}]{
	\IfEq{#1}{0}{\sbra{#2}_0}{\sbra{#2}_{#1}}}


\newcommand{\ointer}[2][{-1}]{\lcrx[#1] ( {#2} ) }
\newcommand{\cinter}[2][{-1}]{\lcrx[#1] [ {#2} ] }
\newcommand{\ocinter}[2][{-1}]{\lcrx[#1] ( {#2} ] }
\newcommand{\cointer}[2][{-1}]{\lcrx[#1] [ {#2} ) }
	

% Set Operations
\newcommand{\PowerSet}{\mathcal{P}}

\newcommand{\union}{\cup}
\newcommand{\Union}{\bigcup}
\newcommand{\djunion}{\sqcup}
\newcommand{\djUnion}{\bigsqcup}

\newcommand{\intersect}{\cap}
\newcommand{\Intersect}{\bigcap}

\newcommand{\vol}{\mathrm{Vol}}
\newcommand{\diam}{\mathrm{diam}}
\DeclareMathOperator{\proj}{proj}

\newcommand{\isomporphic}{\cong}

\DeclareMathOperator{\domain}{dom}
\DeclareMathOperator{\codomain}{codom}

\DeclareMathOperator{\interior}{interior}
\newcommand{\boundary}{\partial}

