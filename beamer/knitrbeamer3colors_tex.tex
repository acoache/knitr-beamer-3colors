\documentclass{beamer}

\makeatletter
\def\maxwidth{ %
	\ifdim\Gin@nat@width>\linewidth
	\linewidth
	\else
	\Gin@nat@width
	\fi
}
\makeatother

% Load packages
\usepackage{graphicx} % Graphics
\usepackage{amsmath,amssymb, mathtools} % Many mathematicals symbols
\usepackage{url} % \url{} command
\usepackage{color} % Colors
\usepackage{natbib} % Citations

% Use the knitr-beamer-3colors theme supplied with this template
\usetheme{knitr-beamer-3colors}
%\usetheme[DarkTheme]{knitr-beamer-3colors}

% Title slide information
\title{Title of the presentation}
\author{Author 1 \and Author 2}
\institute{Department\\Name of the institution}
\titlegraphic{
	\includegraphics[width=3.5cm]{figure-files/placeholder.png}
	\hspace*{1.5cm}
	\includegraphics[width=3.5cm]{figure-files/placeholder.png}
}
\date{\footnotesize{Name of the event -- Date  -- Location of the event}}

\begin{document}
	
	
	% Title slide
	\begin{frame}[plain]
	\maketitle
\end{frame}


% Table of contents
{
	% Set beamer headline to be empty on this slide
	\setbeamertemplate{headline}{
		\leavevmode
		\hbox{
			\begin{beamercolorbox}[wd=\paperwidth,ht=2.5ex,dp=1.125ex]{mblue}
			\end{beamercolorbox}
		}
	}
	
	\begin{frame}{Table of contents}
	
	\begin{enumerate}
		\setlength{\itemsep}{40pt}
		\item Text \\
		\item Block environments
	\end{enumerate}
	
\end{frame}
}


% Section title that will appear in the headline
\section{Text and figures}


% Section frame
\begin{frame}{Text and figures}

\begin{enumerate}
\setlength{\itemsep}{40pt}
\onslide<1>{\item Text and figures} \\
\onslide<0>{\item Block environments}
\end{enumerate}

\end{frame}


% Frame
\begin{frame}[fragile]{List, normal and alerted text}

\begin{itemize}
\setlength{\itemsep}{15pt}
\item The three main colors used in this template are \textcolor{mblue}{blue}, \textcolor{mred}{red} and \textcolor{mgreen}{green}. \\
\item You can easily define new colors and change template colors using the command: \begin{semiverbatim}\\definecolor\{colorname\}\{rgb\}\{0.333, 0.333, 0.333\}\end{semiverbatim}
\item This is some normal text, \alert{alerted text}, \textbf{bold text} and \textit{italic text}. \\
\item This is an url: \href{http://www.anthonycoache.ca}{anthonycoache.ca}.
\end{itemize}

\end{frame}


% Frame
\begin{frame}{Frame with multiple columns}

\begin{minipage}[t]{0.3\linewidth}
	
	With the \texttt{minipage} environment, you can put multiple columns on the same frame.
	
\end{minipage}
\hfill
\begin{minipage}[t]{0.66\linewidth}
	
	This can be useful if you want to comment a figure, a table or R code on the same slide. Be careful with the width of the minipage argument, otherwise you will not obtain two or more distinct columns. The sum of them needs to be less than 1.
	
	\vspace*{0.25cm}
	
	\centerline{\includegraphics[width=0.95\textwidth]{figure-files/placeholder.png}}
	
	
\end{minipage}

\end{frame}


% Frame
\begin{frame}{Figures}

\begin{figure}
\caption{This is an example of a figure, such as graphics and images not generated by R}
\centerline{\includegraphics[width=0.95\textwidth]{figure-files/placeholder.png}}
\end{figure}

\end{frame}


% Frame
\begin{frame}{Tables}

\begin{table}
\caption{This is a table.}
\begin{tabular}{c c c c c}
& \texttt{Sciences} & \texttt{Biology} & \texttt{Whales} & \\
$l_1^\top$ & 1 & 1 & 1 & $\dots$\\
$l_2^\top$ & 0 & 1 & 1 & $\dots$\\
$l_3^\top$ & 1 & 1 & 0 & $\dots$\\
$\vdots$ & $\vdots$ & $\vdots$ & $\vdots$ & $\ddots$ \\
\end{tabular}
\end{table}

\end{frame}


% Section title that will appear in the headline
\section{Block environments}

% Section frame
\begin{frame}{Block environments}

\begin{enumerate}
\setlength{\itemsep}{40pt}
\onslide<0>{\item Text and figures} \\
\onslide<1>{\item Block environments}
\end{enumerate}

\end{frame}


% Frame
\begin{frame}{Environments}

\begin{defnt}{Definition block}
This is a definition block: $$a^2 + b^2 = c^2.$$
\end{defnt}

\begin{mblock}{Normal block}
This is a normal block
\end{mblock}

\begin{thm}{Theorem block}
This is a theorem block: $$a^2 + b^2 = c^2.$$
\end{thm}

\end{frame}


% References
\begin{frame}{References}

With the \texttt{natbib} package, you can either refer to the book of \citet{casella2002statistical} or cite it between parentheses \citep{rosenthal2006struck}. 

\vspace*{0.5cm}

Then this is where you use your \texttt{references.bib} file. You can also add the \texttt{allowframebreaks} chunk options to put references on more than one page. \\

\vspace*{0.5cm}

\footnotesize{
	\bibliographystyle{apalike}
	\bibliography{reference-files/references}
}


\end{frame}


% Last frame
{
\setbeamercolor{background canvas}{use=title, bg=title.bg}
\setbeamercolor{normal text}{use=title, fg=title.fg}
\usebeamercolor[fg]{normal text}

\begin{frame}[plain]

\begin{center}
	\Large{Thank you!}
\end{center}

\vspace*{0.25cm}

\begin{center}
	\textit{Beamer template is available on:} \href{https://github.com/acoache/knitr-beamer-3colors}{\color{title.fg}{github.com/acoache/knitr-beamer-3colors}}
\end{center}


\end{frame}
}

\end{document}
