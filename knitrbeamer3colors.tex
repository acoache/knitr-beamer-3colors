\documentclass{beamer}

\makeatletter
\def\maxwidth{ %
  \ifdim\Gin@nat@width>\linewidth
    \linewidth
  \else
    \Gin@nat@width
  \fi
}
\makeatother

% Load packages
\usepackage{lmodern} % High quality font
\usepackage{graphicx} % Graphics
\usepackage{amsmath,amssymb, mathtools} % Many mathematicals symbols
\usepackage{url} % \url{} command
\usepackage{color} % Colors

% Load theme
\usecolortheme{beaver}

% Define colors
\definecolor{mgreen}{rgb}{0,0.6,0.35}
\definecolor{mblue}{rgb}{0,0.4,0.6}
\definecolor{mred}{rgb}{0.9,0.3,0}
\definecolor{mgrey}{rgb}{0.95,0.95,0.95}
\definecolor{mblack}{rgb}{0,0,0}

% Remove navigation symbols
\setbeamertemplate{navigation symbols}{}

% Set itemize symbols
\setbeamertemplate{enumerate items}{$\bullet$}
\setbeamertemplate{itemize items}{$\bullet$}

% Create a definition environment
\newenvironment{defnt}[1]{
  \setbeamercolor{block body}{bg=mgrey}
  \setbeamercolor{block title}{fg=mblue,bg=mgrey}
  \begin{block}{#1}}{\end{block}}

% Create a theorem environment
\newenvironment{thm}[1]{
  \setbeamercolor{block body}{bg=mgrey}
  \setbeamercolor{block title}{fg=mred,bg=mgrey}
  \begin{block}{#1}}{\end{block}}

% Set colors for text, titles, etc.
\setbeamercolor{title}{fg=mblue}
\setbeamercolor{frametitle}{fg=mblue}
\setbeamercolor{alerted text}{fg=mred}
\setbeamercolor{block title}{fg=mblue}
\setbeamercolor{item projected}{bg=mblue}
\setbeamercolor{item}{fg=mblue}
\setbeamercolor{caption name}{fg=mblue}
\setbeamercolor{normal text}{fg=black!90,bg=white}
\hypersetup{pdfborder={0 0 0},colorlinks=true,linkcolor=mblue,urlcolor=mblue,citecolor=mblue}

% Set beamer headline
\setbeamertemplate{headline}{
	\leavevmode
	\hbox{
		\begin{beamercolorbox}[wd=\paperwidth,ht=2.5ex,dp=1.125ex]{mblue}
			~~~~\insertsection
		\end{beamercolorbox}
	}
}

% Set beamer footline
\setbeamertemplate{footline}[text line]{
	\hfill\strut{
		\scriptsize\sf\color{black!60}
		\quad\insertframenumber
	}
	\hfill
}

% Set beamer transparency of cover / uncover
\setbeamercovered{transparent}


%%%%%


% Title slide information
\title{Title of the presentation}
\author{Author 1 \and Author 2}
\institute{Department\\Name of the institution}
\titlegraphic{
  \includegraphics[width=3.5cm]{placeholder.png}
  \hspace*{1.5cm}
  \includegraphics[width=3.5cm]{placeholder.png}
}
\date{\footnotesize{Name of the event -- Date  -- Location of the event}}

\begin{document}


% Title slide
\begin{frame}[plain]
\maketitle
\end{frame}


% Table of contents
{
	% Set beamer headline to be empty on this slide
	\setbeamertemplate{headline}{
		\leavevmode
		\hbox{
			\begin{beamercolorbox}[wd=\paperwidth,ht=2.5ex,dp=1.125ex]{mblue}
			\end{beamercolorbox}
		}
	}

	\begin{frame}{Table of contents}

	\begin{enumerate}
		\setlength{\itemsep}{40pt}
		\item Text \\
		\item block environments
	\end{enumerate}

  \end{frame}
}


% Section title that will appear in the headline
\section{Text and figures}


% Section frame
\begin{frame}{Text and figures}

\begin{enumerate}
	\setlength{\itemsep}{40pt}
	\onslide<1>{\item Text and figures} \\
	\onslide<0>{\item Block environments}
\end{enumerate}

\end{frame}


% Frame
\begin{frame}{List, normal and alerted text}

\begin{itemize}
  \setlength{\itemsep}{15pt}
  \item The three main colors used in this template are \textcolor{mblue}{blue}, \textcolor{mred}{red} and \textcolor{mgreen}{green}. \\
  \item You can easily define new colors using the command \texttt{\definecolor{mgreen}{rgb}{0,0.6,0.35}} and change template colors. \\
  \item This is some normal text, \alert{alerted text}, \textbf{bold text} and \textit{italic text}. \\
  \item This is an url: \href{http://www.anthonycoache.ca}{anthonycoache.ca}.
\end{itemize}

\end{frame}


% Frame
\begin{frame}{Figures}

\begin{figure}
		\caption{This is an example of a figure, such as graphics and images not generated by R}
		\centerline{\includegraphics[width=0.95\textwidth]{placeholder.png}}
\end{figure}

\end{frame}


% Frame
\begin{frame}{Tables}

\begin{table}
  \caption{This is a table.}
  \begin{tabular}{c c c c c}
    & \texttt{Sciences} & \texttt{Biology} & \texttt{Whales} & \\
    $l_1^\top$ & 1 & 1 & 1 & $\dots$\\
    $l_2^\top$ & 0 & 1 & 1 & $\dots$\\
    $l_3^\top$ & 1 & 1 & 0 & $\dots$\\
    $\vdots$ & $\vdots$ & $\vdots$ & $\vdots$ & $\ddots$ \\
  \end{tabular}
\end{table}

\end{frame}


% Section title that will appear in the headline
\section{Block environments}

% Section frame
\begin{frame}{Block environments}

\begin{enumerate}
	\setlength{\itemsep}{40pt}
	\onslide<0>{\item Text and figures} \\
	\onslide<1>{\item Block environments}
\end{enumerate}

\end{frame}


% Frame
\begin{frame}{Environments}

\begin{defnt}{Definition block}
	This is a definition block: $$a^2 + b^2 = c^2.$$
\end{defnt}

\begin{block}{Normal block}
	This is a normal block
\end{block}

\begin{thm}{Theorem block}
	This is a theorem block: $$a^2 + b^2 = c^2.$$
\end{thm}

\end{frame}


% References
\begin{frame}[plain, fragile]{References}

Create a \texttt{ref.bib} file and add the following commands in this frame: \\

\begin{semiverbatim}
  \\footnotesize\{
	  \\bibliographystyle\{apalike\}
	  \\bibliography\{ref\}
  \}
\end{semiverbatim}


\end{frame}


% Last frame
{
\setbeamercolor{background canvas}{bg=mgrey}

\begin{frame}[plain]

\begin{center}
	\textcolor{mblue}{\Large{Thank you!}}
\end{center}


\end{frame}
}

\end{document}










